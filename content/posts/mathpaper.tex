\documentclass{article}
\usepackage[utf8]{inputenc}
\usepackage{amsmath}

\title{On the nature of kinetic energy and momentum conservation of inelastic and elastic collisons}
\author{Nick Venner}
\date{January 2018}


\begin{document}

\maketitle
\chapterprecishere{``I am being quoted to introduce something, but I have no idea what it is and certainly don't endorse it"\par\raggedleft--- \textup{Randall Munroe}, XKCD}
\begin{abstract}
    The research attempts to model the nature of elastic and inelastic collisions and how energy is conserved when objects collide, our research found that momentum is always conserved through collisions, (when accounting for friction and air resistance), and kinetic energy is always conserved when the collison is elastic and friction is accounted for.
\end{abstract}
\section*{Introduction}
The purpose of this lab is to test the science of collisions, more specifically to test that the speed after a elastic collision is the only solution to these set of equations

\begin{equation}
    {\frac{m_a\cdot v_{a,I
    }^2}{2}+\frac{m_b\cdot v_{b Initial}^2}{2}=\frac{m_a\cdot v_{a Final}^2}{2}+\frac{m_b\cdot v_{b Final}^2}{2}}
    \end{equation}\footnote{Sorry if my equation notation isn't accurate, this is my first time using sub-script notation outside of computer science}

\begin{equation}
    m_a\cdot v_{aInitial}+m_b\cdot v_{b Initial}=m_a\cdot v_{a Final}+m_b\cdot v_{b Final}
\end{equation}
Similarly for inelastic collisions this set of equations hold
\begin{equation}
    m_a\cdot v_{aInitial}+m_b\cdot v_{b Initial}=m_a\cdot v_{a Final}+m_b\cdot v_{b Final}\tag{2}
\end{equation}
\begin{equation}
    v_{a Final}=v_{b Final}
\end{equation}
\section*{Hypothesis,Procedure, Data}
All of these things and more can be found attached to the back of the hypothesis.
\section*{Conclusion}
We found that the data that we collected matched the hypothesis, when friction was accounted for, however we had collected some data of very poor quality, and the data we collected had a very high standard deviation. This suggests a great deal of noise in the detecting mechanism, and inability to isolate other factors. In addition it also suggests that the magnets that we were using in the lab were not perfectly elastic.
\par
A better experiment could probably be done in a vacuum, and having the projectiles hovering on a superconducting magnetic strip. The elastic magnets could be improved by increasing the strength of the magnets. The accuracy of the speed sensor could also be improved by using a laser to  mounted on the track to give an instant continuous reading of position, from witch one could get a much more accurate speed rating.
\end{document}
